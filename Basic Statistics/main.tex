% Mo Jabeen Template for docs 

\documentclass[11pt]{scrartcl} % Font size

\input{structure.tex} % Include the file specifying the document structure and custom commands

%----------------------------------------------------------------------------------------
%	TITLE SECTION
%----------------------------------------------------------------------------------------

\title{	
	\normalfont\normalsize
	\vspace{20pt} % Whitespace
	{\huge Basic Statistics}\\ % The meh
	\vspace{12pt} % Whitespace
	\rule{\linewidth}{2pt}\\ % Thick bottom horizontal rule
}

\author{\small Mo D Jabeen} % Your name

\date{\normalsize\today} % Today's date (\today) or a custom date


\begin{document}

\maketitle % Print the title

\section{Distribution}

\subsection{What is the bernoulli and binomial distribution?}

Bernoulli distribution: The random variable can either be 0 or 1.\\

Binomial distribution: The random variable remains to have only two
states, this shows the probability of measuring either state x number of
times given n independent occurrences.

\section{General}

\subsection{What is the mean, expectation and standard
deviation?}

The mean is the frequency of each value occurring, multiplied by the
value all summed for each random variable.

\begin{equation}
	\overline{x} = 1/n(\Sigma fx)
\end{equation}

The expectation is the probability of each value multiplied by the
value, summed for all values. This is the value the mean tends to as the
sample size increases.

\begin{equation}
	E(x) = \Sigma x P(X=x)
\end{equation}

The variance is the average of the squared difference from the mean.

\begin{equation}
	\sigma^2 = E(X^2) - (E(X))^2
\end{equation}

The standard deviation is the square root of the variance, showing
essentially the average distance from the mean:\(\sigma\). \\

An overall shift to all data points will effect expectation and not
variance:

\begin{equation}
	E(X \pm a) = E(X) \pm a
\end{equation}

\begin{equation}
	Var(X \pm a) = Var(X)
\end{equation}

An overall multiplier to all data points effects both expectation and
variance:

\begin{equation}
	E(aX) = aE(X)
\end{equation}

\begin{equation}
	Var(aX) = a^2Var(X)
\end{equation}

\subsection{Poisson Distribution}

\subsection{What is the poisson distribution
?}

\begin{itemize}
\item The random discrete variable is a count of the number of events occurring
  at random in regions of time and space. Ie radioactive particle emission or saplings in a sample of
  ground

  \begin{itemize}
  \item All events are independent
  \item No two events at the same time
  \item Over a short period of time or on a small region the probability is
    the same
  \end{itemize}
\end{itemize}

\begin{equation}
	p_x = P(X=x) = e^{-\lambda} *\lambda^{x/x!}
\end{equation}

Recurrence formula:

\begin{equation}
	P(X=x) = \lambda/x * P(X= x-1)
\end{equation}

\(\lambda\) is the mean var and \(\sqrt(\lambda)\) is the std

95\% of values are between the mean \(\pm\) 2 std

Independent Poisson random variables can be added to give another
Poisson random variable

\subsection{Continuos variables}

\subsection{What is the difference between continuos and discrete
variables?}

Discrete variables are a known list of possible numbers

Continuos random variables are infinite.

\subsection{What is relative frequency density and how does it
translate to
probability?}

The relative frequency density; is a measurement of the relative
frequency over a class width (interval between two values). Relative frequency is how many times 
something happens between two values compared to number of measurements, 
class width the measurement period.

\begin{equation}
	\frac{Relative frequency}{class width} = Relative frequency density
\end{equation}

The probability density function f(x); is the relative frequency density
as n increases and the class width decreases.\\

Area under a plotted f(x) gives the probability for that range of
continuos variables.

\begin{equation}
	P(X<x) = \int_{-\infty}^{x} f(x)
\end{equation}

\begin{equation}
	\frac{d}{dx}P(X<x) = f(x)
\end{equation}

\subsection{How do you calculate the
Median?}

The median value (m) is when the probability for values above and below
are 0.5.

\begin{equation}
	\int_{m}^{\inf} f(x) = 0.5
\end{equation}

\subsection{How do you calculate a
Percentile?}

The Xth percentile is the value below which the probability is X/100:

90th percentile

\begin{equation}
	P(X<x_{90th}) = 0.9
\end{equation}

\subsection{How do you calculate the
Mean?}

If assume the a small width of delta x, the mean will be
\(\sum x(f(x)\delta(x))\) (the brackets give the probability and
multiplying by x gives the mean). As delta x tends to 0 this becomes:

\begin{equation}
	\bar{x} = \int xf(x)
\end{equation}

The population mean is then :

\begin{equation}
	\mu = \int_{-\infty}^{\infty} xf(x) dx
\end{equation}

The variance is:

\begin{equation}
	Var(x) = \int_{-\infty}^{\infty} (x -\mu)^2 f(x) dx
\end{equation}

\section{Estimation}

\subsection{What is a normal distribution
?}

\textbf{Normal distribution} : Data set centered evenly about a value,
giving a bell curve.

\subsection{How do you calculate the confidence interval
?}

To calculate the confidence interval of a population mean, use the
variable Z below, where \(\bar{X}\) is variable corresponding to the
sample mean:

\begin{equation}
	Z = (\bar{X} - \mu)/ (\sigma/ \sqrt{n})
\end{equation}

For a normal distribution with mean 0 and std of 1 N(0,1) the confidence
interval is:

\begin{equation}
	(\bar{x} - 1.96 (\sigma/\sqrt{n}),\bar{x} - 1.96 (\sigma/\sqrt{n}))
\end{equation}

The above interval on an average of 95\% of the time will include the
mean.

\subsection{What is a T
Distribution?}

If the the sample size is limited and below 30, then instead of a normal
and T distribution is used.

A T dist has degrees of freedom (v) = n-1:

\begin{itemize}
\item A normal distribution has degrees of freedom v = \(\infty\)
\item n being the sample size
\end{itemize}

If the variance is unknown and n is large, Z can be adjusted to use
s\^{}2 which is an unbiased estimate of the variance. This gives two
random variables in the equation X and S:

\begin{equation}
	T = (\bar{X} - \mu)/ (S/\sqrt{n})
\end{equation}

c is the critical value depending on the distribution parameters and the
confidence interval required:

\begin{equation}
	(\bar{x} + c(s\sqrt{n}),\bar{x} - c(s\sqrt{n}))
\end{equation}

\subsection{How is s calculated ?}

\begin{equation}
	s^2 = 1/(n-1) * \sum(x-\bar{x})^2
\end{equation}

\section{Hypothesis testing}

\subsection{What are the two hypothesis
statements?}

2 hypothesis are given the null (\(H_o\)) and the alternative (\(H_1\)):
- Null gives a specific parameter value - Alternative gives a range of
values

Example of a parameter used is the population mean (\(\mu\))

\subsection{How do you determine the confidence of a given null
hypothesis
?}

A normal distribution of N(\(\mu, \sigma^2\)) can be related to N(0,1)
by using \(z = (\bar{x} - \mu)/ (\sigma/\sqrt{n})\).\\

A normal distribution if the variable is adjusted to be the sample mean
\(\bar{X}\) using the mean from the null hypothesis will become
N(\(\mu_o,\sigma^2/n\)).\\

This can then be used in the form:\(P(\bar{X} > x) = P(z > (x-\mu) /\sigma)\) ; 

This is then used to calculate the confidence percentage, from the tails of the distribution
of a N(0,1).\\

\textbf{Two tailed tests gives a much more complete analysis of the data
set.}

\subsection{What are some basic terms
?}

\emph{Test statistic} : Function of data used to determine between
\(H_o\) and\(H_1\)

\emph{The critical region}: Values that lead to rejection of \(H_o\) in
favour of \(H_1\) is the critical region.

\emph{Significance level}: The probability \(H_o\) is rejected for
\(H_1\)

\subsection{What are the error types
?}

\emph{Type 1 error}: \(H_0\) is rejected for \(H_1\) however it was
correct; This is mitigated by choosing a low significance level.

\emph{Type 2 error}: \(H_o\) is accepted but incorrect.

\subsection{What is the suggested test procedure
?}

\begin{enumerate}
\item State the two hypothesis (Null and alternative)
\item Choose the appropriate test statistic and distribution
\item Choose significance level
\item Collect data
\item Analyze
\end{enumerate}

To avoid bias the sig level should be chosen before any data is
collected

If the dataset is approximately normal dist then use the standard N(0,1)

\subsection{How does confidence level relate to significance level
?}

If \(\mu_0\) is outside the range of \(\alpha\)\% confidence level, then
the significance level is (100-\(\alpha\)\%)

\subsection{Chi squared dist}

Uses the standard v degrees of freedom.

Only used for non negative random variables (generally for freq
measurements) to determine if two variables are dependant or
independent. This includes if a variable is bias by comparing it to the
expected non bias result.

\begin{figure}[h] % [h] forces the figure to be output where it is defined in the code (it suppresses floating)
	\centering
	\includegraphics[width=0.5\columnwidth]{Chi-square_pdf.png} % Example image
	\caption{Chi square}
\end{figure}

As shown it is a skewed dist, as v increases the skew decreases.

\subsection{How do you test for
bias?}

The difference between the expected (non bias result) and the observed
result will indicate bias.

Both size and relative size matter:

\begin{equation}
	(O-E) * (O-E)/E = (O-E)^2/E
\end{equation}

O: Observed, E: Expected

\subsection{What value is used to determine goodness of fit between
two
models?}

\begin{equation}
	X^2 = \sum^{m}_{i=1} (O_i - E_i)^2/E_i
\end{equation}

Where m is the number of different outcomes for each model (columns).

\textbf{Large value of \(X^2\) suggest a lack of fit}

\subsection{\texorpdfstring{How does \(X^2\) relate to Chi squared
?}{How does X\^{}2 relate to Chi squared ?}}

Chi squared dist approximately shows the probability distribution of
\(X^2\), if the freq values \textgreater{} 5 :

\begin{equation}
	X^2 = \chi^2_{m-1}
\end{equation}

\subsection{What is a contingency table
?}

A table with more than two variables being measured against (two+ rows)

The degree of freedom is : \(v= (r-1)(c-1)\)

r: rows, c: columns

\textbf{If \(X^2\) is within the chi squared 95\% interval it should be
accepted as independent.}

\subsection{What should you do with a 2x2 table
?}

Can use the alternative:

\begin{equation}
	X^2 = \sum (|O-E|-0.5)^2/E
\end{equation}

\end{document}

%----------------------------------------------------------------------------------------
%	FIGURE EXAMPLE
%----------------------------------------------------------------------------------------

% \begin{figure}[h] % [h] forces the figure to be output where it is defined in the code (it suppresses floating)
% 	\centering
% 	\includegraphics[width=0.5\columnwidth]{IMAGE_NAME.jpg} % Example image
% 	\caption{European swallow.}
% \end{figure}

%----------------------------------------------------------------------------------------
% MATH EXAMPLES
%----------------------------------------------------------------------------------------

% \begin{align} 
% 	\label{eq:bayes}
% 	\begin{split}
% 		P(A|B) = \frac{P(B|A)P(A)}{P(B)}
% 	\end{split}					
% \end{align}

%----------------------------------------------------------------------------------------
%	LIST EXAMPLES
%----------------------------------------------------------------------------------------

% \begin{itemize}
% 	\item First item in a list 
% 		\begin{itemize}
% 		\item First item in a list 
% 			\begin{itemize}
% 			\item First item in a list 
% 			\item Second item in a list 
% 			\end{itemize}
% 		\item Second item in a list 
% 		\end{itemize}
% 	\item Second item in a list 
% \end{itemize}

%------------------------------------------------

% \subsection{Numbered List}

% \begin{enumerate}
% 	\item First item in a list 
% 	\item Second item in a list 
% 	\item Third item in a list
% \end{enumerate}

%----------------------------------------------------------------------------------------
%	TABLE EXAMPLE
%----------------------------------------------------------------------------------------

% \section{Interpreting a Table}

% \begin{table}[h] % [h] forces the table to be output where it is defined in the code (it suppresses floating)
% 	\centering % Centre the table
% 	\begin{tabular}{l l l}
% 		\toprule
% 		\textit{Per 50g} & \textbf{Pork} & \textbf{Soy} \\
% 		\midrule
% 		Energy & 760kJ & 538kJ\\
% 		Protein & 7.0g & 9.3g\\
% 		\bottomrule
% 	\end{tabular}
% 	\caption{Sausage nutrition.}
% \end{table}

%----------------------------------------------------------------------------------------
%	CODE LISTING EXAMPLE
%----------------------------------------------------------------------------------------

% \begin{lstlisting}[
% 	caption= Macro definition, % Caption above the listing
% 	language=python, % Use Julia functions/syntax highlighting
% 	frame=single, % Frame around the code listing
% 	showstringspaces=false, % Don't put marks in string spaces
% 	numbers=left, % Line numbers on left
% 	numberstyle=\large, % Line numbers styling
% 	]

% 	CODE

% \end{lstlisting}

%----------------------------------------------------------------------------------------
%	CODE LISTING FILE EXAMPLE
%----------------------------------------------------------------------------------------

% \lstinputlisting[
% 	caption=Luftballons Perl Script., % Caption above the listing
% 	label=lst:luftballons, % Label for referencing this listing
% 	language=Perl, % Use Perl functions/syntax highlighting
% 	frame=single, % Frame around the code listing
% 	showstringspaces=false, % Don't put marks in string spaces
% 	numbers=left, % Line numbers on left
% 	numberstyle=\tiny, % Line numbers styling
% 	]{luftballons.pl}

%------------------------------------------------

\end{document}